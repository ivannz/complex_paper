Discussion and motivation of CVNN in comparison to double real networks

(Hirose 2009)
Features extensive list of applications of CVNNs (complex valued neural networks):
- ultrasonic fault detection in materials, sonar data, voice processing
- filtering and time-sequential signal processing
- frequency domain mutliplexed microwave signal processing
- pulse beam forming in ultra-wideband communications
- satellite imaging
- image processing
- ground penetrating radar system for landmine detection (buried shallowly)

(merits of CVNN)
Hirose argues that the key merit of CVNN is the built-in (embedded) regularization from
complex arithmetic, esp. multiplication, which does simultaneous phase rotation and amplitude
amplification/attenuation. This reduces ineffective degrees of freedom in comparison
with double-dimensional (paired) real-valued network.

(nature of non-linearities)
argue that cplx non-differentiability is not an issue

Depend on the nature of the signal to be treated wave-related complex signals.
Amplitude-phase type non-linearities

conformal mapping nature of the holomorphic function (mapping structure itself, rather than a combination with some non-linearity)

the non-linearities most promising candidates in CVNNs are usually split, i.e. applied independently to the components of a
complex number, be it real and imaginary (in planar), or phase and amplitude (in polar).


Deep networks achieve state-of-the-art performance on many practical tasks, including
image recognition \citep{citation_needed}, speech synthesis \citep{citation_needed},
automatic translation \citep{citation_needed}. Although performance tends to improve
with the model size, training and deploying large models becomes challenging. Several
aprroaches tackling this are being actively developed: model compression \citep{pruning},
distillation \citep{distill} and sparsification. The latter one essentially reduces
the numerical and storage complexity by zeroing as many parameters of the model as
possbile without significant performance loss. This has foremost importance in resource
constrained applications, such as embedded systems and mobile devices.

sparisity inducing methods impose structure on the topology of sparse weights
unstructured -- gives the achievable maximum of the compression-accuracy tradeoff, but
poses challenges in practical implementations (poor cache performance, almost as if
random access model).

Learnt Sparse topology can be trained from scratch to full performance (this is what
we use in train-compressed phase). reinitialization or continuation. But Gale et al.
failed to replicate successful training from scratch (even for longer training), and
the results were worse than mag-prune. the likely reason is the complexity of the task
and scale of the models.

Sparsification techniques demonstrated deliver sota compression on small models / datasets,
perform inconsistently, and that simple heuristic approaches yield better results with less
computational costs. And provide strong counterexamples to reusability of the learnt sparse
topology.

% their paper
goal: apply sparsity inducing techniques to large-scale models and datasets
\citet{galehooker} raise an important issue that all sparsification techniques have been
proposed and tested on relatively small models, and skew towards convolutional architectures.
Sparsification is like architecture search: the space being the connectivity the computational
graph of the model. Evaluate smooth $\ell_0$ regularization \citep{louizos}, variational
dropout \citep{kingma2015} and magnitude pruning \citep{zhu-gupta2017} \ref{datasets}.

Show that var-dropout and $\ell_0$ demonstrate inconsistent performance on large-scale tasks
in contrast to magnitude pruning, i.e removing small weights $\ell_1$ hard thresholding.

($\ell_0$ regularization) explicitly penalizes the expected number of non-zero parameters
by using multiplicative noise drawn from hard-concrete distribution, that allows differentiating
through sampling.

(var-dropout) var inference over a fully factorized Gaussian posterior over the weights under
log-uniform prior. Local reparameterization to move noise from parameters to intermediate activations
and apply additive noise to further reduce the variance of the gradient. Verify the implementation
by reproducing the results of the original paper \citep{molchanovetal2017}.

(mag-pruning) reactivation of masked weights during training based on grads and sorting schedule.

(random pruning) does not permit reactivation, gives lower bound to sparsity-accuracy trade off.

% datasets and experiment setup
Constant number of training steps, extensive hyperparameter tuning

(random) Benchmark aginst random weight pruning

(resnet) ResNet-50 (He et al. 2016) on ImageNet. 128k iter x 1024 batch 
SGD w. mom; lr ramp-up 5ep. up 0.4, ÷ 10 every 30ep, see TF resnet-50
Sparsify conv and fc layers. Vardropout experiments used batches of size 32 over 32 accel.

(xformer) Neural Machine Translation Transformer (Vasawi et al.2017) on WMT 2014 Eng-Ger dual corpus
standard procedure from (Vasawi et al.2017) without checkpoint averaging. Sparsify fc (attn) and
embedding layers.

% analysis
per layer sparsity and dependence on the hyperparameters: nonuniform sparsity is key to
high compression (He et al. 2018).

(xformer) More param in embedding, FFN an output of attn, less param in key-value of attn.

(resnet) uneven distribution of sparsity through the network

%% in my paper
We use the following three stage-process: train for better initial weights, sparsify
and then fine tune the sparsified model.

these are the techniques for cplx-sparsification. THey allow this trade-off on such and
such models.
